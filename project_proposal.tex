\documentclass[12pt]{article}
\usepackage[utf8]{inputenc}
\usepackage{geometry}
\usepackage{graphicx}
\usepackage{hyperref}
\usepackage{datetime}

\geometry{a4paper, margin=1in}
\newdateformat{monthyeardate}{\monthname[\THEMONTH] \THEYEAR}

\begin{document}

\title{BioPython Toolkit: An Integrated Solution for Gene Sequence Analysis}
\author{COMP 2025 Final Project Proposal}
\date{\monthyeardate\today}
\maketitle

\section*{Team Members}
\begin{itemize}
    \item 
    \item 
    \item 
\end{itemize}

\section*{Abstract}
\begin{flushleft}
The proposed BioPython Toolkit aims to develop an open-source Python package for streamlining gene sequence analysis workflows. This toolkit will integrate essential bioinformatics functions including:
\begin{itemize}
    \item Automated sequence alignment visualization
    \item SNP (Single Nucleotide Polymorphism) detection pipeline
    \item Gene expression quantification module
    \item Interactive phylogenetic tree generator
\end{itemize}

By implementing optimized algorithms and leveraging modern machine learning techniques, this package will provide researchers with a unified interface for common genomic analysis tasks, reducing processing time by up to 40\% compared to existing tools. The toolkit will follow FAIR (Findable, Accessible, Interoperable, Reusable) principles and include comprehensive documentation with Jupyter Notebook examples.
\end{flushleft}

\section*{Project Schedule}
\begin{tabular}{|p{2cm}|p{10cm}|p{2cm}|}
\hline
\textbf{Week} & \textbf{Task} & \textbf{Deadline} \\
\hline
Week 1 & Requirements analysis and literature review & Mar 15 \\
\hline
Week 2-3 & Core architecture design and module prototyping & Apr 1 \\
\hline
Week 4-5 & Implementation of sequence alignment algorithms & Apr 15 \\
\hline
Week 6-7 & Machine learning integration for pattern detection & May 1 \\
\hline
Week 8 & Testing and validation with TCGA dataset & May 15 \\
\hline
Week 9 & Documentation and packaging & May 25 \\
\hline
Week 10 & Final presentation preparation & Jun 1 \\
\hline
\end{tabular}

\end{document}